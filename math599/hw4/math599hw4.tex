\documentclass[12pt]{article}
\usepackage{amssymb,amsmath}
\usepackage{bm}
\newcommand{\eq}[1]{\begin{align*}#1\end{align*}}
\newcommand{\p}[2]{\frac{\partial#1}{\partial#2}}
\newcommand{\de}[2]{\frac{d#1}{d#2}}
\newcommand{\De}[2]{\frac{D#1}{D#2}}
\newcommand{\n}{\bm{\nabla}}
\newcommand{\cross}{\times}
\newcommand{\tc}[3]{#1\times(#2\times#3)}
\newcommand{\mat}[2]{\left[\begin{tabular}{#1}#2\end{tabular}\right]}
\newcommand{\dete}[2]{\left|\begin{tabular}{#1}#2\end{tabular}\right|}
\title{Homework 4}
\author{Jeff Wendling}
\date{November 17th, 2011}
\begin{document}
\maketitle
\section*{Problem 5.1}
If
\eq{
	\bm{x} = (a\cos s + a\alpha t\sin s, a \sin s, 0)
}
Then
\eq{
	\p{}{t}\bm{x} = \bm{u} &= (a\alpha\sin s, 0, 0)\\
	&= (\alpha y, 0, 0)
}
And so at $s = 0$ or $s = \pi$, $\bm{u} = (0, 0, 0)$ and so the particles remain at rest.
Now we calculate
\eq{
	\Gamma &= \int_0^{2\pi} \bm{u}\cdot \p{\bm{x}}{s} ds\\
	&= a^2\alpha \int_0^{2\pi} \sin s(-\sin s + at\cos s)ds\\
	&= f(s)
}
because $\int_0^{2\pi} \sin s \cos s ds = 0$.
\section*{Problem 5.3}
That point of view requires the region to be simply connected because it appeals to Green's theorem.
\section*{Problem 5.4}
We calculate
\eq{
	\de{\Gamma}{t} &= \int_{C(t)} -\frac{1}{\rho}\nabla p \cdot\;d\bm{x}\\
	&= \iint_{D(t)} \nabla \times (-\frac{1}{\rho}\nabla p)\cdot \bm{n}\;d\sigma\\
	&= \iint_{D(t)} -\left[\nabla(\frac{1}{\rho}) \times \nabla p\right]\cdot \bm{n}\;d\sigma\\
	&= \iint_{D(t)} -\left[\frac{\nabla \rho}{\rho^2} \times \nabla p\right]\cdot \bm{n}\;d\sigma\\
	&= \iint_{D(t)} -\left[\frac{1}{\rho^2}(\nabla \rho \times \nabla p)\right]\cdot \bm{n}\;d\sigma
}
But consider that $p = f(\rho)$, then $\nabla p = f'(\rho)\nabla\rho$ and thus $\nabla p$ is in the same direction as $\nabla \rho$, and so $(\nabla \rho \times \nabla p) = 0$, giving
\eq{
	\de{\Gamma}{t} &= \iint_{D(t)} -\left[\frac{1}{\rho^2}(\nabla \rho \times \nabla p)\right]\cdot \bm{n}\;d\sigma\\
	&= \iint_{D(t)} 0 \; d\sigma = 0
}
The unnecessary assumtion was caused by the usage of Stoke's theorem, requiring the contour be simple. We can avoid this by considering
\eq{
	\de{\Gamma}{t} &= \int_{C(t)} -\frac{1}{\rho} \p{p}{s}\;ds\\
	&= \int_{C(t)} -\frac{1}{\rho}f'(\rho)\p{\rho}{s}\;ds\\
	&= \int_{C(t)} -\frac{1}{\rho}f'(\rho)\nabla\rho\cdot d\bm{x}\\
	&= \int_{C(t)} \nabla h \cdot d\bm{x}
}
where
\eq{
	\nabla h &= -\frac{1}{\rho}f'(\rho)\nabla\rho\\
	h &= -\int_0^\rho \frac{1}{s}f'(s)\;ds
}
Thus $h$ is a single valued function, so for a closed contour $C$ we have
\eq{
	\de{\Gamma}{t} &= \int_{C(t)} \nabla h \cdot d\bm{x} = [h] = 0
}
From exercise 1.5 we have
\eq{
	\De{}{t}\left(\frac{\bm{\omega}}{\rho}\right) &= \left(\frac{\bm{\omega}}{\rho} \cdot \nabla\right)\bm{u} + \frac{1}{\rho}\nabla\left(\frac{1}{\rho}\right)\times\nabla p
}
But just as previously, for a barotropic fluid
\eq{
	\frac{1}{\rho}\nabla\left(\frac{1}{\rho}\right)\times\nabla p = 0
}
and so
\eq{
	\De{}{t}\left(\frac{\bm{\omega}}{\rho}\right) &= \left(\frac{\bm{\omega}}{\rho} \cdot \nabla\right)\bm{u}
}
To modify the thin vortex tube argument, we start with a thin vortex
tube in which $\bm{\omega}$ is virtually constant across any
particular cross section. In this case, $\Gamma$ is essentially the
product $\bm{\omega}\delta S$ where $\delta S$ is the normal
cross-section of the tube. But $\delta S$ is also the normal
cross-section of the fluid occupying the tube, and as the fluid must
conserve its mass, $\rho\delta S$ will vary inversely proportional
to the length $l$ of a small section of the tube. Thus, 
$\frac{\bm{\omega}}{\rho}$ varies proportionally to $l$.
So for some constant $c$,
\eq{
	\bm{\omega}\delta S &\approx c\\
	p\delta S &\approx l^{-1}\\
	\frac{\bm{\omega}}{\rho l} &\approx c\\
	\frac{\omega}{\rho} &\approx l
}
\section*{Problem 5.6}
For this problem we start by expanding in tensor notation,
\eq{
	(\nabla\times\bm{a})\times\bm{u} &= e_{nim}(e_{ijk}\p{}{x^j}a_k)u_m\\
	&= -e_{inm}e_{ijjk}\p{}{x^j}a_ku_m\\
	&= -(\delta_{nj}\delta_{mk} - \delta_{nk}\delta_{mj}) \p{}{x^j}a_ku_m\\
	&= \p{}{x^j}a_nu_j - \p{}{x^n}a_ku_k\\
	&= (\bm{u}\cdot\nabla)\bm{a} - \p{a_k}{x^j}u_k
}
Then noticing that because $\nabla \bm{a}\cdot\bm{u}$ being single valued, we have
\eq{
	0 = \int_{C(t)} \nabla \bm{a}\cdot\bm{u}\cdot d\bm{x} = \int_0^1 \p{}{s}(\bm{a}\cdot\bm{u})ds
}
Then,
\eq{
	\de{\mathcal{C}}{t} &= \de{}{t} \int_{C(t)} \bm{a}\cdot d\bm{x}\\
	&= \de{}{t} \int_0^1 \bm{a}\cdot \p{\bm{x}}{s}\;ds\\
	&= \int_0^1 \p{}{t}\left(\bm{a}\cdot\p{\bm{x}}{s}\right)ds\\
	&= \int_0^1 \left(\p{\bm{a}}{t}\right)_s\cdot\p{\bm{x}}{s} + \bm{a}\cdot\p{\bm{u}}{s}ds
}
Now we have
\eq{
	\left(\p{\bm{a}}{t}\right)_s = \left(\p{\bm{a}}{t}\right)_x + (\bm{u}\cdot\nabla)\bm{a}
}
Continuing,
\eq{
	\de{\mathcal{C}}{t} &= \int_0^1 \left( \left(\p{\bm{a}}{t}\right)_x + (\bm{u}\cdot\nabla)\bm{a}\right)\cdot\p{\bm{x}}{s} + \bm{a}\cdot\p{\bm{u}}{s}\;ds\\
	&= \int_0^1 \left( \left(\p{\bm{a}}{t}\right)_x + (\bm{u}\cdot\nabla)\bm{a}\right)\cdot\p{\bm{x}}{s} + \bm{a}\cdot\p{\bm{u}}{s} - \p{}{s}(\bm{a}\cdot\bm{u})\;ds\\
	&= \int_0^1 \left( \left(\p{\bm{a}}{t}\right)_x + (\bm{u}\cdot\nabla)\bm{a}\right)\cdot\p{\bm{x}}{s} - \p{\bm{a}}{s}\cdot\bm{u}\;ds\\
	&= \int_0^1 \left( \left(\p{\bm{a}}{t}\right)_x + (\bm{u}\cdot\nabla)\bm{a}\right)\cdot\p{\bm{x}}{s} - \p{a_i}{x^j}u_i\p{x_j}{s}\;ds\\
	&= \int_0^1 \left( \left(\p{\bm{a}}{t}\right)_x + (\bm{u}\cdot\nabla)\bm{a} - \p{a_i}{x^j}u_i\right)\cdot\p{\bm{x}}{s}\;ds\\
	&= \int_0^1 \left( \left(\p{\bm{a}}{t}\right)_x + (\nabla\times\bm{a})\times\bm{u}\right)\cdot\p{\bm{x}}{s}\;ds\\
	&= \int_{C(t)} \left(\p{\bm{a}}{t} + (\nabla\times\bm{a})\times\bm{u}\right)\cdot d\bm{x}
}
\section*{Problem 5.10}
We consider a line vortex centered at $z_0 = d + iy_0$. The condition that there is no velocity in the x direction on the line $x = 0$ because of a wall leads to placing another line vortex at $z_1 = -d + iy_0$ spinning in the opposite direction. Thus by section 4.4 in the text book, we have
\eq{
	w &= -\frac{i\Gamma}{2\pi}\log(z - z_0) + \frac{i\Gamma}{2\pi}\log(z - z_1)\\
	&= -\frac{i\Gamma}{2\pi}\log(z - d - iy_0) + \frac{i\Gamma}{2\pi}\log(z + d - iy_0)\\
	&= \frac{i\Gamma}{2\pi}\log(\frac{z + d - iy_0}{z - d - iy_0})\\
	&= \frac{i\Gamma}{2\pi}\log(\frac{z + \bar{z_0}}{z - z_0})
}
And we can see that $\p{y_0}{t} = -\frac{\Gamma}{4\pi d}$ by section 5.6 in the textbook (they induce a downward motion on each other, and so they both move down by symmetry).
At the instant $y_0 = 0$, we have on $x = 0$
\eq{
	w &= \frac{i\Gamma}{2\pi}\left( \log(z + \bar{z_0}) - \log(z - z_0)\right)\\
	&= \frac{i\Gamma}{2\pi}\left( \ln(x + d) + i\tan^{-1}\left(\frac{y - y_0}{x + d}\right) - \ln(x - d) - i\tan^{-1}\left(\frac{y - y_0}{x - d}\right) \right)\\
	\\
	v &= \p{\phi}{y} = \frac{\Gamma}{2\pi}\p{}{y}\left( \tan^{-1}\left(\frac{y - y_0}{x - d}\right) - \tan^{-1}\left(\frac{y - y_0}{x + d} \right) \right)\\
	&= \frac{\Gamma}{2\pi}\left( \frac{\frac{1}{x-d}}{1 + \left( \frac{y - y_0}{x -d}\right)^2} - \frac{\frac{1}{x+d}}{1 + \left( \frac{y - y_0}{x + d}\right)^2} \right)\\
	&= \frac{\Gamma}{2\pi}\left( \frac{x-d}{(x-d)^2 + (y-y_0)^2} - \frac{x+d}{(x+d)^2 + (y-y_0)^2} \right)\\
	&= \frac{\Gamma}{2\pi}\left( \frac{-d}{d^2 + y^2} - \frac{d}{d^2 + y^2}\right)\\
	&= \frac{-\Gamma d}{\pi(d^2 + y^2)}
}
And to find $\p{w}{t}$, we define
\eq{
	z + \bar{z_0} &= x + iy + d - iy_0 = r - iy_0\\
	z - z_0 &= x + iy - d - iy_0 = s - iy_0\\
	\\
	r &= x + iy + d\\
	s &= x + iy - d\\
	\\
	m + in &= \frac{z + \bar{z_0}}{z - z_0} = \frac{r - iy_0}{s - iy_0}\\
	\\
	m &= \frac{rs + y_0^2}{s^2 + y_0^2}\\
	n &= \frac{(r-s)y_0}{s^2 + y_0^2}\\
}
Now notice that
\eq{
	w &= \frac{i\Gamma}{2\pi}\log(m + in)\\
	\p{r}{t} &= 0\\
	\p{s}{t} &= 0\\
}
and calculate that when $y_0 \rightarrow 0$,
\eq{
	\p{m}{t} &= \left(\frac{2y_0(s^2 + y_0^2) - (rs + y_0^2)(2y_0)}{(s^2 + y_0^2)^2}\right)y_0' \rightarrow 0\\
	\p{n}{t} &= \left( \frac{(r-s)(s^2 + y_0^2) - ((r-s)y_0)(2y_0)}{(s^2 + y_0^2)^2} \right)y_0' \rightarrow \frac{(r-s)s^2}{s^4}y_0' = \frac{r-s}{s^2}y_0'\\
	m &\rightarrow \frac{r}{s}\\
	n &\rightarrow 0
}
Thus,
\eq{
	\p{w}{t} &= \p{}{t}\frac{-\Gamma}{2\pi}\tan^{-1}\frac{m}{n}\\
	&= -\frac{\Gamma}{2\pi}\left( \frac{\frac{n'm - m'n}{m^2}}{1 + (\frac{n}{m})^2}\right)\\
	&= -\frac{\Gamma}{2\pi}\left( \frac{n'm - m'n}{m^2 + n^2}\right)\\
	&= -\frac{\Gamma}{2\pi}\left( \frac{\left(\frac{r-s}{s^2}y_0'\right)(\frac{r}{s})}{(\frac{r}{s})^2}\right)\\
	&= -\frac{\Gamma(r-s)y_0'}{2\pi rs}\\
	&= -\frac{\Gamma((iy + d) - (iy - d)) \left(\frac{-\Gamma}{4\pi d}\right) }{2\pi (iy + d)(iy - d)}\\
	&= \frac{\Gamma^2(2d)}{8\pi^2 d(-y^2 - d^2)}\\
	&= -\frac{\Gamma^2}{4\pi^2(y^2 + d^2)}
}
So now with those results, we can find the pressure on the wall by starting with
\eq{
	\p{\phi}{t} + \frac{p}{\rho} + \frac{1}{2}\bm{u}^2 + \chi = g(t)\\
}
Choosing $g(t) = \chi$ and noticing that on $x = 0$, the x component of velocity $u = 0$, we have
\eq{
	\p{\phi}{t} + \frac{p}{\rho} + \frac{1}{2}v^2 = 0\\
	p = -\rho\left(\p{\phi}{t} + \frac{1}{2}v^2\right)
}
The pressure is given by
\eq{
	\int_{-\infty}^\infty p dy &= -\rho\int_{-\infty}^\infty\left(\p{\phi}{t} + \frac{1}{2}v^2\right)dy\\
	&= -\rho\int_{-\infty}^\infty\left(-\frac{\Gamma^2}{4\pi^2(y^2 + d^2)} + \frac{-\Gamma^2 d^2}{2\pi^2(d^2 + y^2)^2}\right)dy\\
	&= \rho\frac{\Gamma^2}{\pi^2}\int_{-\infty}^\infty\left(\frac{1}{4(y^2 + d^2)} + \frac{d^2}{2(d^2 + y^2)^2}\right)dy\\
	&= \rho\frac{\Gamma^2}{\pi^2}\frac{\pi}{4d} + \rho\frac{\Gamma^2}{\pi^2}\frac{\pi}{4d}\\
	&= \rho\frac{\Gamma^2}{2\pi d}
}
If the vortex were somehow fixed there would be no pressure change due to the change in the flow potential, thus it's contribution would be gone leaving half the pressure.
\end{document}
