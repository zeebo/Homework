\documentclass[12pt]{article}
\usepackage{amssymb,amsmath}
\usepackage{bm}
\newcommand{\eq}[1]{\begin{align*}#1\end{align*}}
\newcommand{\p}[2]{\frac{\partial#1}{\partial#2}}
\newcommand{\de}[2]{\frac{d#1}{d#2}}
\newcommand{\n}{\bm{\nabla}}
\newcommand{\cross}{\times}
\newcommand{\tc}[3]{#1\times(#2\times#3)}
\newcommand{\mat}[2]{\left[\begin{tabular}{#1}#2\end{tabular}\right]}
\newcommand{\dete}[2]{\left|\begin{tabular}{#1}#2\end{tabular}\right|}
\title{Homework 4}
\author{Jeff Wendling}
\date{November 17th, 2011}
\begin{document}
\maketitle
\section*{Problem 5.1}
If
\eq{
	\bm{x} = (a\cos s + a\alpha t\sin s, a \sin s, 0)
}
Then
\eq{
	\p{}{t}\bm{x} = \bm{u} &= (a\alpha\sin s, 0, 0)\\
	&= (\alpha y, 0, 0)
}
And so at $s = 0$ or $s = \pi$, $\bm{u} = (0, 0, 0)$ and so the particles remain at rest.
Now we calculate
\eq{
	\Gamma &= \int_0^{2\pi} \bm{u}\cdot \p{\bm{x}}{s} ds\\
	&= a^2\alpha \int_0^{2\pi} \sin s(-\sin s + at\cos s)ds\\
	&= f(s)
}
because $\int_0^{2\pi} \sin s \cos s ds = 0$.
\section*{Problem 5.3}
That point of view requires the region to be simply connected because it appeals to Green's theorem.
\section*{Problem 5.4}
We calculate
\eq{
	\de{\Gamma}{t} &= \int_{C(t)} -\frac{1}{\rho}\nabla p \cdot\;d\bm{x}\\
	&= \iint_{D(t)} \nabla \times (-\frac{1}{\rho}\nabla p)\cdot \bm{n}\;d\sigma\\
	&= \iint_{D(t)} -\left[\nabla(\frac{1}{\rho}) \times \nabla p\right]\cdot \bm{n}\;d\sigma\\
	&= \iint_{D(t)} -\left[\frac{\nabla \rho}{\rho^2} \times \nabla p\right]\cdot \bm{n}\;d\sigma\\
	&= \iint_{D(t)} -\left[\frac{1}{\rho^2}(\nabla \rho \times \nabla p)\right]\cdot \bm{n}\;d\sigma
}
But consider that $p = f(\rho)$, then $\nabla p = f'(\rho)\nabla\rho$ and thus $\nabla p$ is in the same direction as $\nabla \rho$, and so $(\nabla \rho \times \nabla p) = 0$, giving
\eq{
	\de{\Gamma}{t} &= \iint_{D(t)} -\left[\frac{1}{\rho^2}(\nabla \rho \times \nabla p)\right]\cdot \bm{n}\;d\sigma\\
	&= \iint_{D(t)} 0 \; d\sigma = 0
}
The unnecessary assumtion was caused by the usage of Stoke's theorem, requiring the contour be simple. We can avoid this by considering
\eq{
	
}
\end{document}