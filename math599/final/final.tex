\title{}
\author{
  \textsc{Jeff Wendling}
}
\date{\today}
\documentclass[11pt]{article}
\usepackage{amsfonts,amsmath,amssymb,amsbsy,amsthm}
\usepackage{latexsym,bm}
\usepackage{upgreek}
\usepackage{graphics,graphicx}
\usepackage{subfigure}
\usepackage{subfigmat}
\usepackage{psfrag}
\usepackage{url}
\usepackage[square,numbers,comma,sort&compress]{natbib}
\usepackage{float}

%macro for unnumbered aligned environments
\newcommand{\eq}[1]{\begin{align*}#1\end{align*}}

%macro for numbered equations
\newcommand{\eqn}[2]{
  \begin{equation}
    \label{#1}
    #2
  \end{equation}
}

%macro for referencing equations
\newcommand{\eqr}[1]{eqn~(\ref{#1})}

%macro for frac partials
\newcommand{\fp}[2]{\frac{\partial#1}{\partial#2}}
\newcommand{\fpp}[2]{\frac{\partial^2 #1}{\partial#2^2}}
\newcommand{\de}[2]{\frac{d #1}{d #2}}
\newcommand{\td}[2]{\frac{D #1}{D #2}}

%macro for inline partials
\newcommand{\p}[1]{\partial_{#1}}
\newcommand{\pp}[1]{\partial^2_{#1}}
\newcommand{\eps}{\epsilon}

\newcommand{\nab}[0]{\bm{\nabla}}

\begin{document}
\begin{center}
{\large\bf 
Jeff Wendling\\
\today\\
Math 528: Elementary Fluid Dynamics I\\
Final
}
\end{center}
\begin{description}
\item[Problem 1] We calculate
\eq{
  (\bm{u}^* \cdot \nab)\bm{u}^*
  &=
  \sqrt{\frac{\rho}{\rho_0}}
  (\bm{u} \cdot \nab)\bm{u}^*
  \\
%
%
  &=
  \sqrt{\frac{\rho}{\rho_0}}
  \big(
  (\bm{u} \cdot \nab)
  (\sqrt{\frac{\rho}{\rho_0}} \bm{u})
  \big)
  \\
%
%
  &=
  \sqrt{\frac{\rho}{\rho_0}}
  \bigg(
  \big(
    (\bm{u} \cdot \nab) u
    \sqrt{\frac{\rho}{\rho_0}}
  \big)
  +
  \underbrace{
  \frac{1}{\sqrt{\rho_0}}
  u
  \big(
    \bm{u} \cdot \nab
  \big)
  ( \sqrt{\rho} )
  }_{=0 \mbox{ (cons of mass)}}
  \bigg)
  \\
%
%
  &=
  \frac{\rho}{\rho_0}
  (-\frac{1}{\rho} \nab p)
  \\
  &=
  -\frac{1}{\rho_0} \nab p
}
\hfill $\blacksquare$
\item[Problem 2]
We solve
$$
  \de{x}{s}(1 + t) = \de{y}{s}
$$
at time $t = 0$, to obtain
$$
  y = x + c
$$
where $c = 0$ because the point $(1, 1)$ is on the line.

To find the trajectory, we solve
\eq{
  \de{x}{t} &= \frac{1}{1+t}
  &\implies
  x &= ln(1 + t) + F_1(\bm{x})
  \\
  \de{y}{t} &= 1
  &\implies
  y &= t + F_2(\bm{x})
}
Becuase the point $(1, 1)$ is on the curve we can solve it to find
$$
  y = e^{x-1}
$$
\hfill $\blacksquare$
\item[Problem 3]
We start by calculating from the given identity
\eq{
  \rho h
  &=
  \rho e + p
  \\
%
%
  \rho \td{h}{t}
  + 
  \underbrace{h}_{= e + \frac{p}{\rho}}
  \td{\rho}{t}
  &= 
  \underbrace{
    \rho \td{e}{t}
  }_{= - p(\nab \cdot \bm{u})}
  + e \td{\rho}{t}
  + \td{p}{t}
  \\
%
%
  \rho \td{h}{t}
  + e \td{\rho}{t}
  + \frac{p}{\rho} \td{\rho}{t}
  &=
  -p(\nab \cdot \bm{u})
  + e\td{\rho}{t}
  + \td{p}{t}
}
Conservation of mass tells us $\td{\rho}{t} = -\rho(\nab \cdot \bm{u})$, thus
after canceling we're left with the result.
$$
  \rho \td{h}{t} = \td{\rho}{t}
$$
\hfill $\blacksquare$
\item[Problem 4]
We use the trig identity
$$
  \sin(2a) + \sin(2b) = 2\cos(a-b)\sin(a+b)
$$
with
  $a = \frac{\pi}{\lambda_1}(x - c_1 t)$
and
  $b = \frac{\pi}{\lambda_2}(x - c_2 t)$
to get the result.

Now, if $\eps = \lambda_2 - \lambda_1$, then we have to a first order approxmation,
\eq{
  \frac{1}{\lambda_1} - \frac{1}{\lambda_2}
  &=
  \frac{1}{\lambda_1} - \frac{1}{\lambda_1 + \eps}
  \\
  &=
  \frac{1}{\lambda_1}
  -
  \frac{1}{\lambda_1}\bigg(
    \frac{1}{1 + \frac{\eps}{\lambda_1}}
  \bigg)
  \\
  &\approx
  \frac{1}{\lambda_1}\bigg(
  1 - (1 + \frac{\eps}{\lambda_1})
  \bigg)
  \\
  &\approx
  \frac{\eps}{\lambda_1^2}
}
And for $c_2 - c_1 = \delta$, we have
\eq{
  \frac{c_1}{\lambda_1} - \frac{c_2}{\lambda_2}
  &=
  \frac{c_1}{\lambda_1} + \frac{c_1 + \delta}{\lambda_1 + \eps}
  \\
  &=
  \frac{c_1}{\lambda_1}
  - \frac{c_1}{\lambda_1 + \eps}
  - \frac{\delta}{\lambda_1 + \eps}
  \\
  &\approx
  c_1(\frac{\eps}{\lambda_1^2})
  -
  \delta(1 + \frac{\eps}{\lambda_1})
  \\
  &\approx
  \frac{c_1\eps}{\lambda_1^2} - \delta
}
Which gives the wavelength as $\lambda_1^2 / \eps$
\hfill $\blacksquare$
\item[Problem 5]
First we calculate
\eq{
  \p{y} &= u\p{\psi} \\
  \pp{y} &= u\p{\psi}u \p{\psi} + u^2 \pp{\psi} \\
  \p{x} &= \p{\eta} - v\p{\psi}
}
Transforming the boundary layer equation with these operators yields
\eq{
  u(\p{\eta}u - v\p{\psi}u)
  +
  v(u\p{\psi} u)
  &=
  -\frac{1}{\rho}
  \p{\eta} p
  +
  \overbrace{
    u\p{\psi} p
  }^{\mbox{small}}
  +
  \nu(
  u (\p{\psi} u)^2 
  + u^2
  \pp{\psi}u
  )
  \\
  \rho u\p{\eta}u
  + \p{\eta}p
  &=
  \rho
  \nu(
  u (\p{\psi} u)^2 
  + u^2
  \pp{\psi}u
  )
  \\
  \p{\eta} h &= \nu \pp{\psi} h
}
where
\eq{
  \p{\eta} h &= \p{\eta}p + \rho(u\p{\eta}u) \\
  \pp{\phi} h &= \rho((\p{\phi} u)^2 + u^2 \pp{\phi}u ) 
}
because $\p{\eta}v = \p{\phi} v = \p{\psi} p $ negligable.
\hfill $\blacksquare$
\item[Problem 6]
I forgot the problem statement :(
\end{description}
\end{document}
