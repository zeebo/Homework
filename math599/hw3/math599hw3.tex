\documentclass[12pt]{article}
\usepackage{amssymb,amsmath}
\usepackage{bm}
\newcommand{\eq}[1]{\begin{align*}#1\end{align*}}
\newcommand{\p}[2]{\frac{\partial#1}{\partial#2}}
\newcommand{\de}[2]{\frac{d#1}{d#2}}
\newcommand{\n}{\bm{\nabla}}
\newcommand{\cross}{\times}
\newcommand{\tc}[3]{#1\times(#2\times#3)}
\newcommand{\mat}[2]{\left[\begin{tabular}{#1}#2\end{tabular}\right]}
\newcommand{\dete}[2]{\left|\begin{tabular}{#1}#2\end{tabular}\right|}
\newcommand{\on}[1]{\operatorname{#1}}
\title{Homework 3}
\author{Jeff Wendling}
\date{November 3rd, 2011}
\begin{document}
\maketitle
\section*{Problem 3.1} Starting with the analysis in section 3.2, and following up to the application of the boundary condition to $f$, we have these equations:
\eq{
	f &= Ce^{ky} + De^{-ky}\\
	\phi &= f(y)\sin(kx - \omega t)\\
	\eta &= A\cos(kx - \omega t)
}
The condition given by finite depth is now $\p{\phi}{y}(-h) = 0$ (there is no vertical velocity at the bottom), or
\eq{
	\p{\phi}{y}(-h) = f'(-h)\sin(kx - \omega t) &= 0\\
	f'(-h) &= 0\\
	Ce^{-kh} &= De^{kh}
}
Thus,
\eq{
	f &= D(e^{2kh}e^{ky} + e^{-ky})\\
	&= d\cosh(k(y + h))
}
So substituting we have
\eq{
	\phi &= d\cosh(k(y + h))\sin(kx - \omega t)\\
	\eta &= A\cos(kx - \omega t)
}
We apply the surface pressure and surface kinematic equations, yielding
\eq{
	\p{\phi}{y} = \p{\eta}{t} &\implies dk\sinh(k(y + h))\sin(kx - \omega t) = A\omega\sin(kx - \omega t)\\
	&\implies dk\sinh(k(y + h)) = A\omega\\
	\\
	\p{\phi}{t} = -g\eta &\implies -\omega d\cosh(ky)\cos(kx - \omega t) = -gA\cos(kx - \omega t)\\
	&\implies \omega d\cosh(k(y + h)) = Ag
}
Multiplying the first equation by $g$, the second by $\omega$, and equating, we obtain
\eq{
	dkg\sinh(k(y + h)) &= \omega^2 d \cosh(k(y + h))\\
	\omega^2 &= kg\tanh(ky + kh)\\
	c^2 &= \frac{g}{k}\tanh(kh)
}
for $y = \eta$ and $\eta << h$, which is the desired result. We must have $\eta << \lambda$ as before, and $\eta << h$ so that we can ignore the $y$ term in the $\tanh$.
For the particle paths, we find the velocity components,
\eq{
	u &= dk\cosh(k(y + h))\cos(kx - \omega t)\\
	v &= dk\sinh(k(y + h))\sin(kx - \omega t)
}
And so the paths follow ellipses that flatten out as $y$ approaches $-h$ ($\sinh(k(y + h)) \rightarrow 0$ as $y \rightarrow -h$).
\section*{Problem 3.2} We begin by considering the pressure condition at the interface where $y = \eta$. Noticing that the pressures are equal at this interface gives the equations (from 3.19)
\eq{
	\rho_1\p{\phi_1}{t} + p + \rho_1\chi &= \rho_1G_1(t)\\
	\rho_2\p{\phi_2}{t} + p + \rho_2\chi &= \rho_2G_2(t)
}
Because $\rho_1$ and $\rho_2$ are constant, we can choose the arbitrary constants to give the pressure condition,
\eq{
	\rho_1\p{\phi_1}{t} + \rho_1g\eta = \rho_2\p{\phi_2}{t} + \rho_2g\eta
}
Continuing as before, we assume
\eq{
	\eta &= A\cos(kx - \omega t)\\
	\phi_1 &= f_1(y)\sin(kx - \omega t)\\
	\phi_2 &= f_2(y)\sin(kx - \omega t)
}
but this time we allow for $k < 0$, and thus
\eq{
	f_1 &= Ce^{|k|y} + De^{-|k|y}\\
	f_2 &= Ee^{|k|y} + Fe^{-|k|y}
}
But we require that $f_1 \rightarrow 0$ as $y \rightarrow -\infty$, and $f_2 \rightarrow 0$ as $y \rightarrow \infty$, thus
\eq{
	\phi_1 &= Ce^{|k|y}\sin(kx - \omega t)\\
	\phi_2 &= Fe^{-|k|y}\sin(kx - \omega t)
}
First we plug these into the surface kinematic condition and obtain
\eq{
	\p{\phi_1}{y} = \p{\eta}{t} &\implies |k|Ce^{|k|y}\sin(kx - \omega t) = \omega A\sin(kx - \omega t)\\
	&\implies |k|Ce^{|k|y} = \omega A\\
	&\implies Ce^{|k|y} = \frac{\omega A}{|k|}\\
	\\
	\p{\phi_2}{y} = \p{\eta}{t} &\implies -|k|Fe^{-|k|y}\sin(kx - \omega t) = \omega A\sin(kx - \omega t)\\
	&\implies -|k|Fe^{-|k|y} = \omega A\\
	&\implies -Fe^{-|k|y} = \frac{\omega A}{|k|}
}
Next we plug into the pressure condition,
\eq{
	\rho_2\p{\phi_2}{t} - \rho_1\p{\phi_1}{t} &= (\rho_1 - \rho_2)g\eta\\
	(-\rho_2 Fe^{-|k|y} + \rho_1Ce^{|k|y})\omega \cos(kx - \omega t) &= (\rho_1 - \rho_2)gA\cos(kx - \omega t)\\
	\omega(-\rho_2 Fe^{-|k|y} + \rho_1Ce^{|k|y}) &= (\rho_1 - \rho_2)gA
}
Using the conditions from the surface kinematic equations,
\eq{
	\omega(-\rho_2 Fe^{-|k|y} + \rho_1Ce^{|k|y}) &= (\rho_1 - \rho_2)gA\\
	\omega(\rho_2 \frac{\omega A}{|k|} + \rho_1 \frac{\omega A}{|k|}) &= (\rho_1 - \rho_2)gA\\
	\frac{\omega^2 A}{|k|}(\rho_2 + \rho_1) &= (\rho_1 - \rho_2)gA\\
	\frac{\omega^2}{k^2} &= \frac{g}{|k|}\frac{\rho_1 - \rho_2}{\rho_1 + \rho_2}\\
	c^2 &= \frac{g}{|k|}\frac{\rho_1 - \rho_2}{\rho_1 + \rho_2}
}
\section*{Problem 3.11}
We begin by considering the disturbance at $t = 0$
\eq{
	\eta(x,0) &= \int_{-\infty}^\infty a(k)e^{ikx}dk\\
	a(k) &= a_0 e^{-\sigma(k - k_0)^2}
}
So with $k_1 = k - k_0$, so $k = k_+1 + k_0$ and $dk = dk_1$,
\eq{
	\eta(x, 0) &= \int_{-\infty}^\infty a_0 e^{-\sigma(k - k_0)^2}e^{ikx}dk\\
	&= \int_{-\infty}^\infty a_0 e^{-\sigma k_1^2}e^{i(k_1 + k_0)x}dk_1\\
	&= a_0 e^{ik_0x}\int_{-\infty}^\infty e^{ik_1 x - \sigma k_1^2} dk_1
}
Notice that
\eq{
	-\sigma k_1^2 + ik_1x &= -\sigma(k_1^2 - \frac{ix}{\sigma}k_1 + \left(\frac{ix}{2\sigma}\right)^2) - \frac{x^2}{4\sigma}\\
	&= -\sigma(k_1 - \frac{ix}{2\sigma})^2 - \frac{x^2}{4\sigma}
}
Thus, letting $z = k_1 - \frac{ix}{2\sigma}$,
\eq{
	a_0 e^{ik_0x}\int_{-\infty}^\infty e^{ik_1 x - \sigma k_1^2} dk_1 &= a_0e^{ik_0 x}e^{-\frac{x^2}{4\sigma}}\int_{-\infty - \frac{ix}{2\sigma}}^{\infty - \frac{ix}{2\sigma}} e^{-\sigma z^2}dz
}
So we evaluate
\eq{
	\int_{-\infty - \frac{ix}{2\sigma}}^{\infty - \frac{ix}{2\sigma}} e^{-\sigma z^2}dz
}
by letting $y = -\frac{x}{2\sigma}$, and considering the closed rectangle in the first two quadrents with corners $(R, R+iy, -R+iy, -R)$ for real $R$. Call $C_1$ the contour from $R$ to $-R$, $C_2$ from $-R$ to $-R+iy$, $C_3$ from $-R+iy$ to $R+iy$, and $C_4$ from $R+iy$ to $R$. 
We attempt to bound the integral over $C_2$ and $C_4$, and find that on $C_2$, $z = -R+it$ for $0 \leq t \leq y$,
\eq{
	|\int_{C_2} e^{-\sigma z^2}| dz &= |i\int_0^y e^{-\sigma (R^2 - t^2 - 2iRt)} dt|\\
	&\leq \int_0^y |e^{-\sigma (R^2 - t^2 - 2iRt)}| dt\\
	&\leq \int_0^y |e^{-\sigma (R^2 - t^2)}| dt\\
	&\leq y|e^{-\sigma R^2}| \rightarrow 0\;\;\on{as}\;\;R\rightarrow\infty
}
A similar result holds for $C_4$, so as $R \rightarrow \infty$, we have that $C_1 = -C_3$ by Cauchy Residue theorem. Thus,
\eq{
	\int_{-\infty - \frac{ix}{2\sigma}}^{\infty - \frac{ix}{2\sigma}} e^{-\sigma z^2}dz &= -\int_{\infty}^{-\infty} e^{-\sigma z^2}dz \\
	&= \int_{-\infty}^{\infty} e^{-\sigma z^2}dz\\
	&= \sqrt{\frac{\pi}{\sigma}}
}
Thus
\eq{
	a_0e^{ik_0 x}e^{-\frac{x^2}{4\sigma}}\int_{-\infty - \frac{ix}{2\sigma}}^{\infty - \frac{ix}{2\sigma}} e^{-\sigma z^2}dz &= a_0e^{ik_0 x}e^{-\frac{x^2}{4\sigma}}\sqrt{\frac{\pi}{\sigma}}
}
as required. Notice that if there are a very large number of crests then $k_0$ is big. If $k_0$ is big then $e^{-\sigma(k - k_0)^2}$ is small unless $k$ is close to $k_0$.
\section*{Problem 3.12} Starting with the equations,
\eq{
	E &= \frac{1}{2}\rho\int_{x_0}^{x_0 + \lambda}\int_{-\infty}^\eta u^2 + v^2 \;dydx + \int_{x_0}^{x_0+\lambda}\int_0^\eta \rho gy \;dydx\\
	\eta &= A\cos(kx - \omega t)\\
	\phi &= \frac{A\omega}{x} e^{ky}\sin(kx -\omega t)\\
	\omega^2 &= gk
}
We have
\eq{
	u &= A\omega e^{ky}\cos(kx - \omega t)\\
	v &= A\omega e^{ky}\sin(kx - \omega t)\\
	\\
	u^2 + v^2 &= A^2\omega^2 e^{2ky}
}
To approximate the kinetic energy term we drop the upper limit of $\eta$ and make it $0$ because this removes only higer order terms, and so
\eq{
	\frac{1}{2}\rho\int_{x_0}^{x_0 + \lambda}\int_{-\infty}^\eta u^2 + v^2 \;dydx &= \frac{\rho A^2\omega^2}{2}\int_{x_0}^{x_0 + \lambda} \int_{-\infty}^0 e^{2ky} \;dydx\\
	&= \frac{\rho A^2\omega^2}{4k}\int_{x_0}^{x_0 + \lambda} 1 \;dx\\
	&= \frac{\rho \lambda A^2 \omega^2}{4k} = \frac{\rho\lambda A^2 g}{4}
}
To find the potential energy term, we evaluate
\eq{
	\int_{x_0}^{x_0+\lambda}\int_0^\eta \rho gy \;dydx &= \rho g \int_{x_0}^{x_0+\lambda}\int_0^\eta y^2 dy\\
	&= \rho g \int_{x_0}^{x_0+\lambda} \frac{1}{2} A^2\cos^2(kx - \omega t) \;dx\\
	&= \frac{A^2 \rho g}{2}\int_{x_0}^{x_0+\lambda}\cos^2(kx - \omega t) \; dx\\
	&= \frac{A^2 \rho g}{2}\int_{x_0}^{x_0+\lambda}(1 + 2\cos(2(kx - \omega t))) \; dx\\
	&= \frac{A^2 \rho g \lambda}{4} + \frac{A^2 \rho g}{4k}\sin(2(kx - \omega t)) \left|_{x_0}^{x_0+\lambda}\right.
}
Now because $k = \frac{2\pi}{\lambda}$,
\eq{
	\sin(2(kx - \omega t)) \left|_{x_0}^{x_0+\lambda}\right. &= \sin(2kx_0 - 2\omega t + 2k\lambda) - \sin(2kx_0 - 2\omega t)\\
	&= \sin(2kx_0 - 2\omega t + 4\pi) - \sin(2kx_0 - 2\omega t)\\
	&= 0
}
Thus,
\eq{
	\int_{x_0}^{x_0+\lambda}\int_0^\eta \rho gy \;dydx &= \frac{A^2 \rho g \lambda}{4}\\
	\\
	\bar{E} = \frac{E}{\lambda} &= \frac{A^2\rho g}{2}
}
From equation (3.19)
\eq{
	\p{\phi}{t} + \frac{p}{\rho} + \frac{1}{2}\bm{u}^2 + \chi = G(t)
}
If we fix an $\bm{x}$, neglect higher order terms, and choose $G(t) = \chi + \frac{p_0}{\rho}$,
\eq{
	\p{\phi}{t} + \frac{p_0 + p_1}{\rho} + \chi &= \chi + \frac{p_0}{\rho}\\
	\p{\phi}{t} + \frac{p_1}{\rho} &= 0\\
	p_1 &= -\rho\p{\phi}{t}\\
}
We then calculate
\eq{
	\int_{-\infty}^\eta p_1 u dy &= \int_{-\infty}^{\eta} -\rho\p{\phi}{t} A\omega e^{ky}\sin(kx - \omega t) \; dy\\
	&= \int_{-\infty}^{\eta} \rho(\frac{A\omega^2}{k}e^{ky}\cos(kx-\omega t)) A\omega e^{ky}\sin(kx - \omega t) \; dy\\
	&= \rho g A^2\omega\cos^2(kx - \omega t)\int_{-\infty}^\eta e^{2ky}dy
}
In which we neglect the contribution from $0$ to $\eta$ and find,
\eq{
	\int_{-\infty}^0 e^{2ky}dy &= \frac{1}{2k}
}
Thus,
\eq{
	\int_{-\infty}^\eta p_1 u dy &= \frac{\rho g A^2\omega}{2k}\cos^2(kx - \omega t)
}
To find the average potential, we integrate over one time period $\frac{2\pi}{\omega}$, and divide by it to find the average, and find
\eq{
	\bar{F} &= \frac{\omega}{2\pi} \frac{\rho g A^2\omega}{2k}\int_{0}^{\frac{2\pi}{\omega}} \cos^2(kx - \omega t) \; dt\\
	&= \frac{\rho g A^2\omega^2}{8k\pi}\int_{0}^{\frac{2\pi}{\omega}} (1 + \cos(2(kx - \omega t))) \; dx\\
	&= \frac{\rho g A^2\omega^2}{8k\pi}(t - \frac{1}{2\omega}\sin(2(kx - \omega t)))\left|_{0}^{\frac{2\pi}{\omega}}\right.\\
	&= \frac{\rho g A^2\omega}{4k} = \frac{\rho g A^2 c}{4}
}
Now,
\eq{
	\frac{\bar{F}}{\bar{E}} &= \frac{\frac{\rho g A^2 c}{4}}{\frac{A^2\rho g}{2}}\\
	&= \frac{\rho g A^2 c}{4}\frac{2}{A^2\rho g}\\
	&= \frac{c}{2}\\
}
And (3.29) says that $c_g = \frac{d\omega}{dk}$, thus
\eq{
	\omega^2 = gk\\
	2\omega\frac{d\omega}{dk} &= g\\
	\frac{d\omega}{dk} &= \frac{1}{2}\frac{g}{\omega}
}
But $\omega^2 = gk \implies \frac{g}{\omega} = \frac{\omega}{k}$ and $c = \frac{\omega}{k}$, so
\eq{
	\frac{d\omega}{dk} = \frac{1}{2}\frac{g}{\omega} = \frac{c}{2}
}
\end{document}