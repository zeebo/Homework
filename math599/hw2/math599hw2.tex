\documentclass[12pt]{article}
\usepackage{amssymb,amsmath}
\usepackage{bm}
\newcommand{\eq}[1]{\begin{align*}#1\end{align*}}
\newcommand{\p}[2]{\frac{\partial#1}{\partial#2}}
\newcommand{\de}[2]{\frac{d#1}{d#2}}
\newcommand{\n}{\bm{\nabla}}
\newcommand{\cross}{\times}
\newcommand{\tc}[3]{#1\times(#2\times#3)}
\newcommand{\mat}[2]{\left[\begin{tabular}{#1}#2\end{tabular}\right]}
\newcommand{\dete}[2]{\left|\begin{tabular}{#1}#2\end{tabular}\right|}
\title{Homework 2}
\author{Jeff Wendling}
\date{October 20, 2011}
\begin{document}
\maketitle
\section*{Problem 6.1} With the convention that $[\bm{x}]_i$ is the $i$th
component of the vector $\bm{x}$, we notice
\eq{
	[2(\bm{n} \cdot \n)\bm{u}]_i &= 2n_j\p{u_i}{x_j}
}
and
\eq{
	[\tc{\bm{n}}{\n}{\bm{u}}]_i &= e_{ijk}e_{klm}n_j\p{}{x_l}u_m\\
	&= e_{kij}e_{klm}n_j\p{}{x_l}u_m\\
	&= [\delta_{il}\delta_{jm} - \delta_{im}\delta_{jl}]n_j\p{}{x_l}u_m\\
	&= n_j\p{u_j}{x_i} - n_j\p{u_i}{x_j}
}
Thus
\eq{
	[2(\bm{n} \cdot \n)\bm{u} + \bm{n} \cross (\n \cross \bm{u})]_i &= 2n_j\p{u_i}{x_j} + n_j\p{u_j}{x_i} - n_j\p{u_j}{x_j}\\
	&= n_j\p{u_i}{x_j} + n_j\p{u_j}{x_i}\\
	&= n_j(\p{u_i}{x_j} + \p{u_j}{x_i})
}
Which means
\eq{
	t_i &= -pn_i + n_j(\p{u_i}{x_j} + \p{u_j}{x_i})
}
\section*{Problem 6.2} Starting with equation (6.13) and integrating over the suface of some arbitrary volume,
\eq{
	\bm{t} &= -p\bm{n} + \mu[2(\bm{n}\cdot\n)\bm{u} + \tc{\bm{n}}{\n}{\bm{u}}]\\
	\int_S \bm{t} \;dS &= \int_S -p\bm{n} + \mu[2(\bm{n}\cdot\n)\bm{u} + \tc{\bm{n}}{\n}{\bm{u}}] \;dS\\
	&= \int_V -\n p \;dV + \mu \int_S 2(\bm{n}\cdot\n)\bm{u} + \tc{\bm{n}}{\n}{\bm{u}} \;dS\\
	&= \int_V -\n p + 2\mu\n^2 \bm{u} \;dV + \mu\int_S \tc{\bm{n}}{\n}{\bm{u}} \;dS
}
Now consider (because of incompressability)
\eq{
	\int_V \n^2 \bm u \;dV &= \int_V \n(\n\cdot\bm{u}) - \tc{\n}{\n}{\bm{u}} \;dV\\
	&= -\int_V \tc{\n}{\n}{\bm{u}} \;dV\\
	&= \int_S (\n \times \bm{u}) \times \bm{n} \;dV\\
	&= -\int_S \tc{\bm{n}}{\n}{\bm{u}} \;dV
}
Thus
\eq{
	\int_S \bm{t} \;dS &= \int_V -\n p + 2\mu\n^2 \bm{u} \;dV + \mu\int_S \tc{\bm{n}}{\n}{\bm{u}} \;dS\\
	&= \int_V -\n p + 2\mu\n^2 \bm{u} \;dV - \mu\int_V \n^2 \bm u \;dV\\
	&= \int_V -\n p + \mu\n^2 \bm{u} \;dV
}
\section*{Problem 6.3}
For the case of simple shear flow, $\bm{u} = [u(y), 0, 0]$ and when $\bm{n} = [0, 1, 0]$ we have
\eq{
	-p\bm{n} &= [0, -p, 0]\\
	\\
	2(\bm{n}\cdot\n)\bm{u} &= 2(\p{}{y})[u(y), 0, 0]\\
	&= [2\de{u}{y}, 0, 0]\\
	\\
	\n\cross\bm{u} &= \dete{ccc}{
		x & y & z \\
		$\p{}{x}$ & $\p{}{y}$ & $\p{}{z}$ \\
		u(y) & 0 & 0 \\
	}\\
	&= [0, 0, -\de{u}{y}]\\
	\tc{\bm{n}}{\n}{\bm{U}} &= \dete{ccc}{
		x & y & z \\
		0 & 1 & 0 \\
		0 & 0 & $-\de{u}{y}$ \\
	}\\
	&= [-\de{u}{y}, 0, 0]
}
Thus by equation 6.13 we have
\eq{
	\bm{t} &= -p\bm{n} + \mu[2(\bm{n}\cdot\n)\bm{u} + \tc{\bm{n}}{\n}{\bm{u}}]\\
	&= [0, -p, 0] + \mu\left([2\de{u}{y}, 0, 0] + [-\de{u}{y}, 0, 0]\right)\\
	&= [2\mu\de{u}{y}, -p, 0]
}
\section*{Problem 6.9}
For the first method we use direct substitution into 6.13 for $\bm{u} = u_r(r, \theta)\bm{e_r}$. Note that in this case $\bm{n} = \bm{e_\theta}$ because it is the outward normal to the lower part of the channel. Also $\n = [\p{}{r} ,\frac{1}{r}\p{}{\theta}]$ and $\bm{n}\cdot\n = \frac{1}{r}\p{}{\theta}$.\\
First we cast 6.13 into an appropriate form using
\eq{
	\tc{\bm{n}}{\n}{\bm{u}} &= \n(\bm{u}\cdot\bm{n}) - \tc{\bm{u}}{\n}{\bm{n}} - (\bm{u}\cdot\n)\bm{n} - (\bm{n}\cdot\n)\bm{u}\\
	&= \n(\bm{u}\cdot\bm{n}) - (\bm{n}\cdot\n)\bm{u}
}
But in this case $\bm{u}\perp\bm{n}$, so
\eq{
	\tc{\bm{n}}{\n}{\bm{u}} &= - (\bm{n}\cdot\n)\bm{u}
}
Then
\eq{
	2(\bm{n}\cdot\n)\bm{u} + \tc{\bm{n}}{\n}{\bm{u}} &= (\bm{n}\cdot\n)\bm{u}
}
Then 6.13 becomes
\eq{
	\bm{t} &= -p\bm{n} + \mu(\bm{n}\cdot\n)\bm{u}\\
	&= -p\bm{e_\theta} + \frac{\mu}{r}\p{}{\theta} u_r\bm{e_r}\\
	&= -p\bm{e_\theta} + \frac{\mu}{r}\p{u_r}{\theta}\bm{e_r} + \frac{\mu u_r}{r}\bm{e_\theta}\\
	&= \frac{\mu}{r}\p{u_r}{\theta}\bm{e_r} + \left(-p + \frac{\mu u_r}{r}\right)\bm{e_\theta}
}
For the second method we calculate using the definition of the tensor vector in terms of the stress tensor,
\eq{
	t_i &= T_{ij} n_j
}
Thus because $\bm{n} = n_r\bm{e_r} + n_\theta\bm{e_\theta} = 0\bm{e_r} + 1\bm{e_\theta}$, our vector satisfies
\eq{
	t_i &= T_{i\theta}
}
Then we can find
\eq{
	t_r &= T_{r\theta}\\
	&= -p\delta_{r\theta} + 2\mu e_{r\theta}\\
	&= \mu\left( r\p{}{r}(\frac{u_\theta}{r}) + \frac{1}{r}\p{u_r}{\theta} \right)\\
	&= \frac{\mu}{r}\p{u_r}{\theta}
}
and
\eq{
	t_\theta &= T_{\theta\theta}\\
	&= -p\delta_{\theta\theta} + 2\mu e_{\theta\theta}\\
	&= -p + 2\mu\left( \frac{1}{r}\p{u_\theta}{\theta} + \frac{u_r}{r} \right)\\
	&= -p + \frac{2\mu u_r}{r}
}
\section*{Problem 6.13}
We write the Jacobian of
\eq{
	\mat{ccc}{
		$\p{x_1}{X_1}$ & $\p{x_1}{X_2}$ & $\p{x_1}{X_3}$ \\
		\\
		$\p{x_2}{X_1}$ & $\p{x_2}{X_2}$ & $\p{x_2}{X_3}$ \\
		\\
		$\p{x_3}{X_1}$ & $\p{x_3}{X_2}$ & $\p{x_3}{X_3}$ \\
	}
}
in tensor notation as
\eq{
	J = e_{ijk}\p{x_1}{X_i}\p{x_2}{X_j}\p{x_3}{X_k}
}
Notice that for Lagrangian coordinates, $\frac{D}{Dt} = \p{}{t}$, so
\eq{
	\frac{DJ}{Dt} &= \p{J}{t}\\
	&= \p{}{t}e_{ijk}\p{x_1}{X_i}\p{x_2}{X_j}\p{x_3}{X_k}\\
	&= e_{ijk} \p{u_1}{x_m} \p{x_m}{X_i}\p{x_2}{X_j}\p{x_3}{X_k} + \\
	& \;\;\;\;e_{ijk} \p{u_2}{x_m} \p{x_m}{X_i}\p{x_3}{X_j}\p{x_1}{X_k}\\
	& \;\;\;\;e_{ijk} \p{u_3}{x_m} \p{x_m}{X_i}\p{x_1}{X_j}\p{x_2}{X_k}\\
	&\equiv h^1_{ijk} + h^2_{ijk} + h^3_{ijk}
}
Now consider
\eq{
	h^1_{ijk} &= e_{ijk} \p{u_1}{x_m} \p{x_m}{X_i}\p{x_2}{X_j}\p{x_3}{X_k}\\
	&= \p{u_1}{x_1} e_{ijk}\p{x_1}{X_i}\p{x_2}{X_j}\p{x_3}{X_k} + \p{u_1}{x_2} e_{ijk}\p{x_2}{X_i}\p{x_2}{X_j}\p{x_3}{X_k} + \p{u_1}{x_3} e_{ijk}\p{x_3}{X_i}\p{x_2}{X_j}\p{x_3}{X_k}
}
but the last two terms in the sum are clearly zero because
\eq{
	e_{ijk} = -e_{jik}\\
	e_{ijk} = -e_{kji}
}
So
\eq{
	h^1_{ijk} &= \p{u_1}{x_1} e_{ijk}\p{x_1}{X_i}\p{x_2}{X_j}\p{x_3}{X_k}\\
	h^1 &= \p{u_1}{x_1} J
}
Similarly
\eq{
	h^2 &= \p{u_2}{x_3} J\\
	h^3 &= \p{u_2}{x_3} J
}
Thus
\eq{
	\frac{DJ}{Dt} &= h^1 + h^2 + h^3\\
	&= J(\p{u_1}{x_1} + \p{u_2}{x_2} + \p{u_3}{x_3})\\
	&= J(\n \cdot \bm{u})
}

To prove Reynold's transport theorem, we consider the change of some quantity of some volume as it is transported or deformed through time,
\eq{
	\de{}{t}\int_{V(t)} G dV &= \de{}{t} \int_{V(t)} G dx_1dx_2dx_3\\
	&= \de{}{t} \int_{V(0)} GJ dX_1 dX_2 dX_3
}
which we can now interchange integration and differentiation because the limits no longer depend on time, giving
\eq{
	\de{}{t} \int_{V(0)} GJ dX_1 dX_2 dX_3 &= \int_{V(0)} \p{}{t} (GJ) dX_1 dX_2 dX_3\\
	&= \int_{V(0)} \left[\p{G}{t}J + G\p{J}{t}\right] dX_1 dX_2 dX_3
}
But we have that $\p{J}{t} = \frac{DJ}{Dt} = J(\n \cdot \bm{u})$ and inside of these transformed coordinates $\p{G}{t} = \frac{DG}{Dt}$, so
\eq{
	\int_{V(0)} \left[\p{G}{t}J + G\p{J}{t}\right] dX_1 dX_2 dX_3 &= \int_{V(0)} \left[J\frac{DG}{Dt} + GJ(\n \cdot \bm{u})\right] dX_1 dX_2 dX_3\\
	&=\int_{V(0)} \left[\frac{DG}{Dt} + G(\n \cdot \bm{u})\right]J dX_1 dX_2 dX_3\\
	&= \int_{V(t)} \frac{DG}{Dt} + G(\n \cdot \bm{u}) dx_1 dx_2 dx_3
}
Proving the theorem.
\end{document}