\title{Investigation into some numeric solutions to Volterra Integral equations of the first kind}
\author{
\textsc{Jeff Wendling}
}
\date{\today}
\documentclass[11pt]{article}
\usepackage{amsfonts,amsmath,amssymb,amsbsy}
\usepackage{latexsym,bm}
\usepackage{upgreek}
\usepackage{graphics,graphicx}
\usepackage{subfigure}
\usepackage{subfigmat}
\usepackage{psfrag}
\usepackage{url}
\usepackage[square,numbers,comma,sort&compress]{natbib}
\usepackage{listings}
\usepackage{lstlang0}
\lstset{language=Go, numbers=left, showspaces=false}
\numberwithin{equation}{section}
\newtheorem{theorem}{Theorem}[section]
\newtheorem{definition}[theorem]{Definition}
\newtheorem{lemma}[theorem]{Lemma}
\newtheorem{corollary}[theorem]{Corollary}
\newtheorem{fact}[theorem]{Fact}
\newtheorem{example}[theorem]{Example}

\begin{document}
\maketitle
\begin{abstract}
I present a simple numerical scheme for evaluation volterra integral
equations of the first kind. I prove some simple results about
convergence and verify the results numerically. These results are
compared to a method by Lubich in cases where the domain is large.
\end{abstract}
\setcounter{tocdepth}{1}
\tableofcontents
\lstlistoflistings
%\listoffigures
%\listoftables
\section{Examples}
\lstset{caption=Basic Scheme}
\lstinputlisting{/Users/zeebo/Code/volterra/conv/conv.go}
\lstset{caption=Lubich Scheme}
\lstinputlisting{/Users/zeebo/Code/volterra/lubich/lubich.go}
\section{Results}
some tables of results
\end{document}