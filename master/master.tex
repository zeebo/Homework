\title{
	\textsc{Investigation Into Some Numeric Solutions to Volterra Integral Equations of the First Kind}
}
\author{
	\textsc{Jeff Wendling}
}
\date{\today}
\documentclass[11pt]{article}
\usepackage{amsfonts,amsmath,amssymb,amsbsy,amsthm}
\usepackage{latexsym,bm}
\usepackage{upgreek}
\usepackage{graphics,graphicx}
\usepackage{subfigure}
\usepackage{subfigmat}
\usepackage{psfrag}
\usepackage{url}
\usepackage[square,numbers,comma,sort&compress]{natbib}
\usepackage{listings}
\usepackage{lstlang0}
\usepackage{float}

%dont reposition tables
\restylefloat{table}
\lstset{language=Go, numbers=left, showspaces=false}
\numberwithin{equation}{section}
\theoremstyle{definition}
\newtheorem{theorem}{Theorem}[section]
\newtheorem{definition}[theorem]{Definition}
\newtheorem{lemma}[theorem]{Lemma}
\newtheorem{corollary}[theorem]{Corollary}
\newtheorem{fact}[theorem]{Fact}
\newtheorem{example}[theorem]{Example}

%% some macros

%macro for unnumbered aligned environments
\newcommand{\eq}[1]{\begin{align*}#1\end{align*}}

%macro for numbered equations
\newcommand{\eqn}[2]{
	\begin{equation}
		\label{#1}
		#2
	\end{equation}
}

%macro for referencing equations
\newcommand{\eqr}[1]{Equation (\ref{#1})}

\begin{document}
\maketitle
\begin{abstract}
I present a simple numerical scheme for evaluation volterra integral
equations of the first kind. I prove some simple results about
convergence and verify the results numerically. These results are
compared to a method by Lubich in cases where the domain is large.
\end{abstract}
\setcounter{tocdepth}{1}
\tableofcontents
\lstlistoflistings
%\listoffigures
\listoftables
\section{Description of Method}
We start with a convolution type equation
\eqn{base_convolution}{
	K \star x = f
}
and pick some upper time $T$ and discritize the interval $[0, T]$ with $N$
equally spaced points, $t_0 = 0, ..., t_N = T$. We approximate the solution $x$
as a linear combination of indicator functions $\chi_i$ where
\eqn{indicator_defn}{
	\chi_i(t)
	=
	\left\{
	\begin{array}{ll}
		1 & t_{i} < t \leq t_{i+1} , \\
		0 & \text{otherwise}
	\end{array}
	\right.
}
giving
\eqn{solution_approx}{
	x_N(t) = \sum_{i=0}^{N-1} c_i \chi_i(t)
}
Substituting \eqr{solution_approx} into \eqr{base_convolution} and cosidering
the value of the new convolution at time $t_n$, we find that
\begin{align}
\nonumber            	(K \star x_N)(t_n)
                     		&= \int_0^{t_n} K(t_n - s)x(s)\; ds \\
\nonumber            		&= \int_0^{t_n} K(t_n - s) \sum_{i=0}^{N-1} c_i \chi_i(s)\; ds \\
\nonumber            		&= \int_0^{t_n} K(t_n - s) \sum_{i=0}^{n-1} c_i \chi_i(s)\; ds \\
\label{scheme_before}		&= \sum_{i=0}^{n-1} c_i \int_{t_i}^{t_{i+1}} K(t_n - s)\; ds
\end{align}
Now notice that with a simple substitution $u = t_n - s$
\eqn{kernel_substitution}{
	\int_{t_i}^{t_{i+1}}K(t_n - s)\; ds = \int_{t_{n-i-1}}^{t_{n-i}} K(u) du
}
which leads to the definition
\eqn{kernel_integral}{
	K_i = \int_{t_i}^{t_{i+1}} K(s)\; ds
}
Using \eqr{kernel_substitution} and \eqr{kernel_integral} into \eqr{scheme_before}
gives
\begin{align}
\nonumber     	\sum_{i=0}^{n-1} c_i \int_{t_i}^{t_{i+1}} K(t_n - s)\; ds
              		&= \sum_{i=0}^{n-1} c_i \int_{t_{n-i-1}}^{t_{n-i}} K(u)\; du\\
\label{scheme}		&= \sum_{i=0}^{n-1} c_i K_{n-i-1} = f_N(t_n)
\end{align}
where we define $f_N$ to agree with $f$ at every point $t_i$ for $i \in 1, ..., N$.
The inner points depend on the kernel $K$ so that we have the relation
\eqn{approx_relation}{
	K \star x_N = f_N
}
\eqr{scheme} leads us to a numerical scheme for solving the convolution. It is
an iterative scheme in that given the values $c_0, ..., c_{n-1}$ we can find
the value $c_n$ by
\eqn{specific_term}{
	c_nK_0 = f(t_{n+1}) - \sum_{i=0}^{n-1} c_i K_{n-i}
}
\section{Proof of Convergence}
In this section we prove convergence for the scheme given by \eqr{specific_term}.
We begin by stating some theorems about convolutions,
\begin{theorem}
	\label{convolution_exists}
	$f \star g$ exists if $f \in L^1$, $g \in L^1$. When this is the case, $f \star g \in L^1$
\end{theorem}
\begin{proof}
The proof is given by Theorem 1.3 in \cite{stein71}
\end{proof}

\begin{theorem}
	\label{titchmarsh}
	If for $f \in L^1$, $g \in L^1$, $f \neq 0$ almost everywhere on $(0, T)$, and $f \star g = 0$ almost everywhere on $(0, T)$, then $g = 0$ almost everywhere on $(0, T)$.
\end{theorem}
\begin{proof}
This is a special case of the Titchmarsh convolution theorem given in \cite{titchmarsh}
\end{proof}

With those under our belt, we now discuss convergence. From 
\section{Sample Code}
\lstset{caption=Basic Scheme}
\lstinputlisting{conv/conv.go}
\lstset{caption=Lubich Scheme}
\lstinputlisting{lubich/lubich.go}
Please refer to table \ref{some_table} for some results.
\section{Numerical Results}
Some results are listed below.
\begin{table}[H]
	\caption{Error in $L^2$ vs stepsize}
	\label{some_table}
	\begin{center}
	\begin{tabular}{|c|c|}
	\hline
	Stepsize & Error \\ \hline
	.1 & $6e^{-7}$ \\ \hline
	.01 & $6e^{-8}$ \\ \hline
	.001 & $6e^{-9}$ \\ \hline
	.0001 & $6e^{-10}$ \\
	\hline
	\end{tabular}
	\end{center}
\end{table}
foo bar
\begin{table}[H]
	\caption{Error in $L^2$ vs stepsize}
	\label{some_table_2}
	\begin{center}
	\begin{tabular}{|c|c|}
	\hline
	Stepsize & Error \\ \hline
	.1 & $6e^{-7}$ \\ \hline
	.01 & $6e^{-8}$ \\ \hline
	.001 & $6e^{-9}$ \\ \hline
	.0001 & $6e^{-10}$ \\
	\hline
	\end{tabular}
	\end{center}
\end{table}
:()
\begin{table}[H]
	\caption{Error in $L^2$ vs stepsize}
	\label{some_table_3}
	\begin{center}
	\begin{tabular}{|c|c|}
	\hline
	Stepsize & Error \\ \hline
	.1 & $6e^{-7}$ \\ \hline
	.01 & $6e^{-8}$ \\ \hline
	.001 & $6e^{-9}$ \\ \hline
	.0001 & $6e^{-10}$ \\
	\hline
	\end{tabular}
	\end{center}
\end{table}
Always put a blank space after a table.

\bibliographystyle{alpha}
\bibliography{master}
\end{document}